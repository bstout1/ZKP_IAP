\documentclass[12pt,letterpaper]{article}
\usepackage[latin1]{inputenc}
\usepackage{amsmath, amsthm, amssymb, amsfonts, mathrsfs, amscd, fancyheadings}
\pagestyle{fancy}
\usepackage[top=1in, left=1in, right=1in]{geometry}% http://ctan.org/pkg/geometry
\usepackage{lipsum}% http://ctan.org/pkg/lipsum

\def\AA{{\mathbb A}}
\def\BB{{\mathbb B}}
\def\CC{{\mathbb C}}
\def\FF{{\mathbb F}}
\def\GG{{\mathbb G}}
\def\KK{{\mathbb K}}
\def\MM{{\mathbb M}}
\def\NN{{\mathbb N}}
\def\OO{{\mathbb O}}
\def\QQ{{\mathbb Q}}
\def\PP{{\mathbb P}}
\def\QQ{{\mathbb Q}}
\def\RR{{\mathbb R}}
\def\TT{{\mathbb T}}
\def\ZZ{{\mathbb Z}}

\def\Asf{{\mathsf A}}
\def\Bsf{{\mathsf B}}
\def\Esf{{\mathsf E}}
\def\Fsf{{\mathsf F}}
\def\Psf{{\mathsf P}}
\def\Ssf{{\mathsf S}}

\def\hhat{{\hat h}}
\def\Hhat{{\hat H}}

\def\0{{\mathbf 0}}
\def\1{{\mathbf 1}}
\def\a{{\mathbf a}}
\def\b{{\mathbf b}}
\def\c{{\mathbf c}}
\def\e{{\mathbf e}}
\def\h{{\mathbf h}}
\def\r{{\mathbf r}}
\def\w{{\mathbf w}}
\def\x{{\mathbf x}}
\def\y{{\mathbf y}}
\def\z{{\mathbf z}}
\def\Abf{{\mathbf A}}
\def\Bbf{{\mathbf B}}

\def\Acal{{\mathcal A}}
\def\Bcal{{\mathcal B}}
\def\Ccal{{\mathcal C}}
\def\Dcal{{\mathcal D}}
\def\Ecal{{\mathcal E}}
\def\Fcal{{\mathcal F}}
\def\Gcal{{\mathcal G}}
\def\Hcal{{\mathcal H}}
\def\Ical{{\mathcal I}}
\def\Jcal{{\mathcal J}}
\def\Kcal{{\mathcal K}}
\def\Lcal{{\mathcal L}}
\def\Mcal{{\mathcal M}}
\def\Ncal{{\mathcal N}}
\def\Ocal{{\mathcal O}}
\def\Pcal{{\mathcal P}}
\def\Qcal{{\mathcal Q}}
\def\Rcal{{\mathcal R}}
\def\Scal{{\mathcal S}}
\def\Tcal{{\mathcal T}}
\def\Vcal{{\mathcal V}}
\def\Xcal{{\mathcal X}}
\def\Zcal{{\mathcal Z}}

\def\pfrak{{\mathfrak p}}
\def\Gfrak{{\mathfrak G }}
\def\Rfrak{{\mathfrak R }}

\def\Kbar{{\bar K}}
\def\kbar{{\bar k}}
\def\Fbar{{\bar F}}

\def\arch{\mathrm{arch}}
\def\Aut{\mathrm{Aut}}
\def\ch{\mathrm{char}}
\def\Cl{\mathrm{Cl}}
\def\ker{\mathrm{Ker}}
\def\Div{\mathrm{Div}}
\def\diag{\mathrm{diag}}
\def\Perm{\mathrm{Perm}}
\def\diam{\mathrm{diam}}
\def\disc{\mathrm{disc}}
\def\sgn{\mathrm{sgn}}
\def\grad{\mathrm{grad}}
\def\div{\mathrm{div}}
\def\Isom{\mathrm{Isom}}
\def\ns{\mathrm{ns}}
\def\ab{\mathrm{ab}}
\def\s{\mathrm{s}}
\def\Gal{\mathrm{Gal}}
\def\Nor{\mathrm{Nor}}
\def\Sym{\mathrm{Sym}}
\def\sym{\mathrm{sym}}
\def\Rat{\mathrm{Rat}}
\def\Hom{\mathrm{Hom}}
\def\Spec{\mathrm{Spec}}
\def\Vol{\mathrm{Vol}}
\def\PGL{\mathrm{PGL}}
\def\GL{\mathrm{GL}}
\def\tor{\mathrm{tor}}
\def\Pic{\mathrm{Pic}}
\def\PrePer{\mathrm{PrePer}}
\def\Per{\mathrm{Per}}
\def\Crit{\mathrm{Crit}}
\def\Fix{\mathrm{Fix}}
\def\ord{\mathrm{ord}}
\def\ad{\mathrm{ad}}
\def\Res{\mathrm{Res}}
\def\Prep{\mathrm{Prep}}
\def\trace{\mathrm{trace}}
\def\norm{\mathrm{norm}}
\def\Berk{\mathrm{Berk}}
\def\supp{\mathrm{supp}}
\def\st{\mathrm{st}}
\def\im{\mathrm{Im}}
\def\CS{\mathrm{CS}}
\def\min{\mathrm{min}}

\def\GLn{\mathrm{GL_n}}
\def\Frac{\mathrm{Frac}}
\def\uf{\mathrm{uf}}
\def\uc{\mathrm{uc}}
\def\Tw{\mathrm{Twist}}

\theoremstyle{plain}
\newtheorem{thm}{Theorem}
\newtheorem{prb}{Problem}
\newtheorem{conj}{Conjecture}
\newtheorem{cor}[thm]{Corollary}
\newtheorem{prop}[thm]{Proposition}
\newtheorem{lem}[thm]{Lemma}

\theoremstyle{definition}
\newtheorem*{dfn}{Definition}
\newtheorem*{thmdfn}{Theorem/Definition}
\newtheorem*{dfns}{Definitions}
\newtheorem*{rem}{Remark}
\newtheorem{ex}{Example}

\begin{document}

\title{ZKP IAP: Session 3 Homework}
\author{Brian Justin Stout\footnote{email: bjstout@proton.me}}
\maketitle

\begin{prb}\label{P1}
Prove the quadratic non-residue interactive protocol is complete and sound.
\end{prb}
Suppose $m, x$ are positive integers. Assume that $QR(m,x)=0$, i.e. that x is not a quadratic residue, mod $m$. Assume the prover is honest. Let $s\in\ZZ_m$ and $b$ be arbitrary. If $b=0$, then $y=s^2x$. In this case $y$ is also not a quadratic residue mod $m$ (if it were, then $x$ would be too), so Prover sends back $0$, which matches $b$.

If $b=1$, then Verifier sends $y=s^2$. This is a quadratic residue, so $QR(y,m)=1$ and Prover sends back $1$, matching $b$. The protocol is complete.

To show soundness, Verifier selects $b$ as $0$ or $1$ with probability of $\dfrac{1}{2}$ each. Lets say that Prover responsds with $0$ with probability $p$ and $1$ with probability $1-p$. Then the probability verifier rejects is $\dfrac{1}{2}p+\dfrac{1}{2}(1-p)=\dfrac{1}{2}$. 

\begin{prb}\label{P2}
Prove the quadratic residue interactive protocol is complete, sound, knowledge sound, and zero knowledge.
\end{prb}

The proof of soundness and completeness are identical to the previous problem.

To prove knowledge soundness, assume that the Verifier could query both $b=0$ and $b=1$ at the same time. In this case, Verifier receives both $t$ and $st$. He recovers $s=t*t^{-1}$, so Prover does indeed know $s$.

To prove zero knowledge we argue as follows. Half of integers modulo $m$ are quadratic resiudes. Verifier learns the following during his interaction with Prover: $y=xt^2$. Because $t^2$ is a quadratic resiude, the map $y\mapsto yt^2$ is a permutation of the set of resides and non-residues, respectively. Verifier then learns either $(0,t)$ or $(1,st)$ with probabilities of $\dfrac{1}{2}$ and $\dfrac{1}{2}$. So with probability $\dfrac{1}{2}$ Verifier knows $x, m, xt^2, 0, t$ and with probability $\dfrac{1}{2}$ Verifier knows $x, m, xt^2, 1, st$. The distribution of these elements is identical.

\begin{prb}\label{P3}
Suppose a group G has an efficiently computable nondegenerate bilinear self pairing. Give an efficient algorithm for deciding given $\alpha G, \beta G, H\in\GG$ whether $\alpha\beta G = H$.
\end{prb}
Let $e$ be the pairing. Check $e(\alpha G, \beta G) = e(H, G)$

\begin{prb}\label{P4}
Check the BLS signature scheme accepts a correctly signed signaure. Argue it is computationally infeasible to to find a forged signature of any message $m$ if the forger is given $pk$ but not $sk$. What are come computational hardness assumptions?
\end{prb}

Assume $\sigma$ is a a correct signature for a message $m$. Then
$$e(g_0,\sigma)=e(g_0,\alpha H(m))=e(\alpha g_0, H(m))=e(pk,H(m))$$
This requires the DDH assumption for the pairing and groups. It also requires a secure hash function. For any message $m$ to create a forged signature requires solving the discrete log problem for $\GG_0$, which is assumed hard.
\end{document}
